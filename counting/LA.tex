\section{与线性代数有关的一些计数定理}
\subsection{矩阵树定理}
对于一个无向图$G$,假定其中点用$1\ldots n$编号,这些点的度数分
别为$d_1,\ldots,d_n$,我们定义$G$的度数矩阵为:
\[D=\text{diag}(d_1,\ldots,d_n)\]
同时我们假定$G$的邻接矩阵为$A$,最后定义$G$的基尔霍夫矩阵$Q=D-A$
,那么$Q$的任一余子式$M_{ii}$(将第$i$行与第$i$列删后的矩阵的行列
式)即为原图生成树个数。

\subsection{Lindström–Gessel–Viennot 引理}
对于一个 DAG $G$,假定其起点集合为$A={a_1,\ldots,a_n}$,终点
集合为$B={b_1,\ldots,b_n}$,设矩阵
\[M=\begin{bmatrix}
  e(a_1, b_1) & e(a_1, b_2) & \ldots & e(a_1, b_n)\\
  e(a_2, b_1) & e(a_2, b_2) & \ldots & e(a_2, b_n)\\
  \vdots      & \vdots      & \ddots & \vdots     \\
  e(a_n, b_1) & e(a_n, b_2) & \ldots & e(a_n, b_n)
\end{bmatrix}\]
其中$e(x, y)$表示从$x$到$y$的路径数。\par
定义「合法方案」为一种将$A$与$B$完美匹配,然后每一对匹配在$G$上对应
一条路径,这些路径两两不相交(无公共点)。\par
那么:
\[\det(M)=\sum_{(P_1,P_2,\ldots,P_n):A\to B}\text{sgn}(\sigma(P))\]

\subsection{行列式和完美匹配存在性}
对于图$E$定义如下的 Edmonds 矩阵$A$:
\[A_{ij}=\begin{cases}
  x_{ij} & (i,j)\in E\\
  0      & (i,j)\notin E
\end{cases}\]
需要注意,这里的$x_{ij}$是两两不同的变量!\par
显然,$|A|$是一个多元多项式,且$|A|$的项数也就是完美匹配的数量。
但这一方法是不甚实用的:要去求这种多元多项式的行列式也只能$\Theta(n^2)$
来做。不过由 Schwartz-Zippel 定理可知,如果我们把所有变元都带
以随机值\footnote{这里假设对一个大质数取模。},那么误判率是很
小的。