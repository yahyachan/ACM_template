\section{阶乘幂和斯特林数}
\subsection{阶乘幂}
\[x^{\underline k} = \frac{x!}{(x - k)!}\]\par
\[x^{\overline k} = \frac{(x + k - 1)!}{(x - 1)!}\]\par
第一个式子是不是有点亲切呢?对比一下:
\[\binom x k = \frac{x!}{k!(x - k)!}\]
\subsection{斯特林数}
\subsubsection{定义}
定义第一类斯特林数$n\brack k$表示将$n$个元素划分为$k$个环的方案数。通俗
点说,就是有$k$个环的$1\ldots n$的排列的方案数。\par
定义第二类斯特林数$n\brace k$表示将$n$个本质不同的物品划分为$k$个无标号
非空集合的方案数(或者说划分成$k$)个等价类也行。\par
多扯一句,有一个贝尔数$B_n$表示了将$n$个本质不同元素划分成若干等价类的
方案数。
\subsubsection{递推式}
\[{n\brack k} = {{n - 1}\brack {k - 1}} + (n - 1){{n - 1}\brack k}\]\par
\[{n\brace k} = {{n - 1}\brace {k - 1}} + k{(n - 1)\brace k}\]
\subsubsection{与阶乘幂的关系}
\[x^k = \sum_{i = 0}^k{k\brace i}x^{\underline i}
        = \sum_{i = 0}^k(-1)^{k - i}{k\brace i}x^{\overline i}\]\par
\[x^{\overline k} = \sum_{i = 0}^k{k\brack i}x^i\]
\[x^{\underline k} = \sum_{i = 0}^k(-1)^{k - i}{k\brack i}x^i\]\par
一言以蔽之,第一类是阶乘幂转幂,第二类是幂转阶乘幂,因此第二类更常用一点……
\subsubsection{斯特林反演}
\[{n\brace k} = \frac{1}{k!}\sum_{i = 0}^k(-1)^{k-i}\binom k ii^n\]
\subsubsection{高效求某一行的斯特林数}
第一类斯特林数怎么求?注意到$x^{\overline k}$就是第$k$行第一类斯特林数
的OGF。\par
第二类斯特林数怎么求?看上一页,然后卷积那一套东西优化。
\subsubsection{例题}
试推:
\[\sum_{i = 0}^n\binom n i i^k\]\par
把常规幂转成下降幂,有:
\[
\begin{aligned}
\quad&\sum_{i = 0}^n\binom n i\sum_{j = 0}^k{k\brace j}j!\binom i j\\
=&\sum_{j = 0}^k{k\brace j}j!\sum_{i = 0}^n\binom n i\binom i j\\
=&\sum_{j = 0}^k{k\brace j}j!\binom n j\sum_{i = 0}^n\binom{n - j}{i - j}\\
=&\sum_{j = 0}^k{k\brace j}j!\binom n j\sum_{i = j}^n\binom{n - j}{i - j}\\
=&\sum_{j = 0}^k{k\brace j}j!\binom n j2^{n - j}
\end{aligned}
\]
\section{差分与牛顿级数}
\subsection{差分与高阶差分}
\subsubsection{差分}
\[\Delta^0f(x) = f(x)\]
\[\Delta^kf(x) = \Delta^{k - 1}f(x + 1) - \Delta^{k - 1}f(x)\]\par
类似于微分$\mathrm d$,差分$\Delta$是一个算子,也是可以参与运算的……
\subsubsection{高阶差分求法}
定义另一个算子$\mathrm Ef(x) = f(x + 1)$,然后我们发现$\Delta = \mathrm E - 1$
,带入高阶差分得到:
\[
\begin{aligned}
\Delta^kf(x) &= (E - 1)^kf(x)\\
&= \sum_{i = 0}^k\binom k i (-1)^{n - i}E^if(x)\\
&= \sum_{i = 0}^k\binom k i (-1)^{n - i}f(x + i)
\end{aligned}
\]\par
如果$x$是固定的,那么很显然这个东西可以FFT优化吧(
\subsection{牛顿级数}
任何$k$次多项式$f(x)$都可以展开为组合数的形式,具体的说是:
\[f(x) = \sum_{i = 0}^k\Delta^if(0)\binom x i\]