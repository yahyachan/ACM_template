\section{反射法}
\subsection{一维反射法}
\paragraph{题面}
数轴上有$1\ldots n$这$n$个点,给定一个起点$s$,每次可以向左或向右走
一步,整个过程中坐标不可以$<1$或$>n$。求恰好走了$i$步到达点$d$的方案
数。
\paragraph{解析}
首先我们不考虑越不越界,把答案加上随便走的答案(这个利用组合数不难算出
)。越界的话,一定擦到过$0$或者$n+1$处。若是擦到$0$,若将第一次擦到$0$
后运动都反向,那么终点即变成$d$关于关于$0$的对称点$P(d)=-d$,从答案中
减去到$P(d)$的方案数;至于擦到$n+1$的情况,也做类似操作,终点变成关于
$n+1$的对称点$Q(d)=2(n+1)-d$,同理从答案中减去到$Q(d)$的方案数。需要
注意可能多次擦到边界,因此需要多次翻转容斥。

\subsection{二维反射法}
\paragraph{题面}
在二维平面上,从原点出发,每次只能向右或向上走,不得越过$y=x+a$和$y=x-b$
两条直线\footnote{只有一条限制的情况会更简单,不需要多次翻转。},求走
到某一特定终点的方案数。
\paragraph{解析}
首先不考虑限制的话,可以直接用组合数求出一个从原点走到终点的方案数。如果
考虑限制,先考虑穿过了$y=x+a$的情况,这样一定向上碰到了$y=x+a+1$。我们
将从原点$O$到第一次碰触点这一段路径翻转一下(右、上两种操作对换),那么
就把这类不合法路径转化为了从新原点到终点的方案数。穿过$y=x-b$的情况是类
似的。\par
这里还是可能违反多种限制,因此要多次翻转容斥。