\chapter{数论}

\section{解方程}
\subsection{首一多项式的有理根}
最高次项系数为$1$的整系数多项式的有理根一定是整数。\par
这也就变相告诉我们,代数整数和有理数的交集是整数集。

\section{积性函数的前缀和}
\subsection{MIN25筛}
(下面假设$f$为求和对象)\par
首先预处理$g(i, x)$表示$[1, x]$内为质数或与前$i$个质数都互质的答案,每次只需要对
$x\geq p_i^2$的情况递推,柿子为:
\[g(i, x) = g(i - 1, x) - f(p_i)(g(i - 1,\lfloor\frac{x}{p_i}\rfloor) - g(i - 1, p_i - 1))\]\par
然后用$f(i, x)$表示$x$范围内与前$i-1$个质数都互质的答案,则最终答案为$f(1)+f(1, n)$。
先把质数的答案都算进来(之前预处理了),然后考虑合数的答案。考虑枚举所有不小于$p_i$的质数$p$(假设是第$j$
个质数),如果说$p^2>x$了那么就没法往下转移了(因为这样用上$p$就没法构造合数了),break出来就行了
;反之则枚举$p$在数中所占的正指数$e$,对于所有$p^{e+1}\leq x$,对答案做
$f(j + 1, \lfloor\frac{x}{p^e}\rfloor)f(p^e) + f(p^{e + 1})$的贡献(
其实就是枚举是否选够了$e$个$p$,然后考虑只用$p$的若干次方的情况)。\par
下面给出对$\sigma^3$(即$\sigma^3(x) = \sum_{d | x}d^3$)求和的代码,有一些细节
需要根据具体求和函数更改。
\lstinputlisting{num_thy/prefix_sum/min_25.cpp}