\chapter{概率论}
\section{随机游走}
最一般的随机游走当然是图上高斯消元了,要是是 DAG 就直接 DP。
但还是有不少特殊情况可以优化的。
\lstinputlisting{probability/random_walk/loj2542.cpp}
\subsection{树上随机游走}
叶子节点的状态可以表示为关于父亲状态的一次函数,往上一路推到
根上就可。
\subsection{网格图上的随机游走}
\subsubsection{直接消元法}
直接每次只消去与当前活跃的方程相邻的方程。
\lstinputlisting{probability/random_walk/hdu6994.cpp}
\subsubsection{主元法}
只设第一行(或者第一列)的变量,其他变量用这些来做线性表出。\par
不过需要注意这个方法在可能有零概率转移的情形下复杂度会变高,
当然好处是可以适用于「类似」网格图的图。
\subsection{最一般情况(多段,中间可能重复走)}
(这里假设图强连通)\par
设$f_{i, j}$表示从$i$走到$j$的期望步数,$p_{i, k}$表示边
$(i, k)$被走的概率,$g_i$表示从$i$走出去再走回来的期望步
数。用大写字母表记这些量对应矩阵,而且用$J$表示全$1$矩阵,
那么转移表示为:
\[(I - P)F = J - G\]\par
我们再引入稳态分布$\pi$(一个行向量),使得$\sum_{i=1}^n \pi_{i} = 1$,
且$\pi P = \pi$。要求$\pi$,只需要注意到$\sum_{i}\pi_i = 1$
是一个显然成立的方程,然后$\pi P = \pi$也即$\pi(P - I) = 0$
又给出了$n$个方程,高斯消元解方程即可。由于$\pi$满足如下性质:
\[\pi_i g_i = 1\]
所以马上就可以求出$G$。\par
此时还有另一问题就是$I-P$不满秩。我们将$F$的方程记为$AX=B$,
强行消元,得到:
\[
\begin{bmatrix}
  1 & 0 & 0 & \ldots & 0 & a_1\\
  0 & 1 & 0 & \ldots & 0 & a_2\\
  0 & 0 & 1 & \ldots & 0 & a_3\\
  \vdots & \vdots & \vdots & \ddots & \vdots & \vdots\\
  0 & 0 & 0 & \ldots & 1 & a_{n-1}\\
  0 & 0 & 0 & \ldots & 0 & 0
\end{bmatrix}X=B'
\]
这时候强行求出方程一组特解$Y$,就有$X_{i,j} = Y_{i,j} - Y_{j,j}$。\par
这种 useless algorithm 谁出谁傻逼好吧(大嘘)
\lstinputlisting{probability/random_walk/frank.cpp}