\chapter{杂项}
\section{矩阵的特征多项式}
这里给出一个$O(n^3)$的矩阵特征多项式求法,不过在需要输出浮点数的场合
可能数值稳定性不佳。\par
首先我们肯定是要化成一个相似意义下较为标准的形式,但上三角形式并不一
定能简单的化出来。因此我们每次用第$i+1$行作为主元$i$的代表行来进行相
似消元,从而得到一个相似于原矩阵的、只有下次对角线和上三角区域非零的
矩阵(称上 Hessenberg 矩阵)$A$。\par
设矩阵前$i$行前$i$列组成矩阵的特征多项式为$p(i)$,那么有如下递推关系:
\[p_i(x)=(x-A_{i,i})p_{i-1}(x)-\sum_{m=1}^{i - 1}A_{i - m,i}(\prod_{j=1}^m A_{i-j+1,i-j}) p_{i - m - 1}(x)\]
直接暴力就行。
\lstinputlisting{sundry/charpoly.cpp}

\section{多项式矩阵的行列式}
假设矩阵中多项式次数至多为$d$,矩阵行列数为$n$,则下面给出$O((nd)^3)$
的算法。\par
我们可以根据多项式系数将我们要求的多项式写作:
\[\det(M_0+xM_1+\ldots+x^{d}M_d)\]
上述行列式等价于:
\[\begin{vmatrix}
  xI_n & -I_n & 0 & \ldots & \ldots & 0\\
  0 & xI_n & -I_n & \ldots & \ldots & 0\\
  \vdots & \vdots & \vdots & \ddots & \vdots & \vdots\\
  0 & 0 & 0 & \ldots & xI_n & -I_n\\
  M_0 & M_1 & M_2 & \ldots & M_{d-1} & M_d
\end{vmatrix}\]
首先不妨假设$M_d$满秩,这样我们可以对最后几行做初等行变换来将矩阵化为如下形式:
$$
\begin{vmatrix}
xI_n & -I_n & 0 & \ldots & \ldots & 0\\
0 & xI_n & -I_n & \ldots & \ldots & 0\\
\vdots & \vdots & \vdots & \ddots & \vdots & \vdots\\
0 & 0 & 0 & \ldots & xI_n & -I_n\\
M'_0 & M'_1 & M'_2 & \ldots & M'_{d-1} & I_n
\end{vmatrix}
$$
将分块矩阵最后一行加上倒数第二行,得到:
$$
\begin{vmatrix}
xI_n & -I_n & 0 & \ldots & \ldots & 0\\
0 & xI_n & -I_n & \ldots & \ldots & 0\\
\vdots & \vdots & \vdots & \ddots & \vdots & \vdots\\
0 & 0 & 0 & \ldots & xI_n & -I_n\\
M'_0 & M'_1 & M'_2 & \ldots & M'_{d-1}+xI_n & 0
\end{vmatrix}
$$
对最后一列做 Laplace 展开,行列式变成:
$$
\begin{vmatrix}
xI_n & -I_n & 0 & \ldots & 0\\
0 & xI_n & -I_n & \ldots & 0\\
\vdots & \vdots & \vdots & \ddots & \vdots\\
M'_0 & M'_1 & M'_2 & \ldots & M'_{d-1}+xI_n
\end{vmatrix}
$$
这样一来,就化出了一个特征多项式的形式,用上一节中算法即可。\par
如果$M_d$不满秩呢?我们来想一下对其做高斯消元的过程,假设在
第$i$​列找不到$\ge i$​的行的非$0$​元,这说明目前对应的多项式矩
阵$M_0+xM_1+\ldots+x^{d}M_d$​​中第$i$​列所有$\ge i$​的行的次
数都小于$d$​。为了进行进一步操作,我们先要将$<i$​行上第$i$​列的
次数变成小于$d$​​的,具体来说就是利用对角线上的多项式去消除第$i$
列。这样操作后第$i$列所有多项式次数都小于$d$了,我们先将第$i$
列整体乘$x$(最后算答案的时候要除回来),再看能否找到$\ge i$
的行中次数等于$d$的多项式……若如此反复多次(假设补上的$x$已经
超过$nd$了那肯定不合理)扔不能找到,那么答案就为$0$​了。
\lstinputlisting{sundry/poly_det.cpp}

\section{向量集(二进制分组维护动态加点凸包及其上的三分)}
前半部分是 Andrew 算法求凸包,后半部分是二进制分组维护动态加点凸包。
\lstinputlisting{sundry/loj2197.cpp}